%%%%%%%%%%%%%%%%%%%%%%%%%%%%%%%%%%%%%%%%%%%%%%%%%%%%%%%%%%
%
% Vzor pro sazbu kvalifikační práce
%
% Západočeská univerzita v Plzni
% Fakulta aplikovaných věd
% Katedra informatiky a výpočetní techniky
%
% Petr Lobaz, lobaz@kiv.zcu.cz, 2016/03/14
%
%%%%%%%%%%%%%%%%%%%%%%%%%%%%%%%%%%%%%%%%%%%%%%%%%%%%%%%%%%

% Možné jazyky práce: czech, english
% Možné typy práce: BP (bakalářská), DP (diplomová)
\documentclass[czech,BP]{thesiskiv}

% Definujte údaje pro vstupní strany
%
% Jméno a příjmení; kvůli textu prohlášení určete, 
% zda jde o mužské, nebo ženské jméno.
\author{Kateřina Kratochvílová}
\declarationfemale

% Název práce
\title{Automatická anotace\\obrázků}

% 
% Texty abstraktů (anglicky, česky)
%
\abstracttexten{The text of the abstract (in English). It contains the English translation of the thesis title and a short description of the thesis.}

\abstracttextcz{V této bakalářské práci se zabíváme automatickou anotací obrázků..
}

% Na titulní stranu a do textu prohlášení se automaticky vkládá 
% aktuální rok, resp. datum. Můžete je změnit:
%\titlepageyear{2016}
%\declarationdate{1. března 2016}

% Ve zvláštních případech je možné ovlivnit i ostatní texty:
%
%\university{Západočeská univerzita v Plzni}
%\faculty{Fakulta aplikovaných věd}
%\department{Katedra informatiky a výpočetní techniky}
%\subject{Bakalářská práce}
%\titlepagetown{Plzeň}
%\declarationtown{Plzni}

%%%%%%%%%%%%%%%%%%%%%%%%%%%%%%%%%%%%%%%%%%%%%%%%%%%%%%%%%%
%
% DODATEČNÉ BALÍČKY PRO SAZBU
% Jejich užívání či neužívání záleží na libovůli autora 
% práce
%
%%%%%%%%%%%%%%%%%%%%%%%%%%%%%%%%%%%%%%%%%%%%%%%%%%%%%%%%%%

% Zařadit literaturu do obsahu
\usepackage[nottoc,notlot,notlof]{tocbibind}

% Umožňuje vkládání obrázků
\usepackage[pdftex]{graphicx}

%matematicke znaky

% Odkazy v PDF jsou aktivní; navíc se automaticky vkládá
% balíček 'url', který umožňuje např. dělení slov
% uvnitř URL
\usepackage[pdftex]{hyperref}
\hypersetup{colorlinks=true,
  unicode=true,
  linkcolor=black,
  citecolor=black,
  urlcolor=black,
  bookmarksopen=true}

% Při používání citačního stylu csplainnatkiv
% (odvozen z csplainnat, http://repo.or.cz/w/csplainnat.git)
% lze snadno modifikovat vzhled citací v textu
\usepackage[numbers,sort&compress]{natbib}

%%%%%%%%%%%%%%%%%%%%%%%%%%%%%%%%%%%%%%%%%%%%%%%%%%%%%%%%%%
%
% VLASTNÍ TEXT PRÁCE
%
%%%%%%%%%%%%%%%%%%%%%%%%%%%%%%%%%%%%%%%%%%%%%%%%%%%%%%%%%%
\begin{document}
%
\maketitle
\tableofcontents

\chapter{Úvod}
V dnešní době, kdy je svět přesycen obrázky v digitální podobě, není vůbec snadné nalést obrázek zobrazující požadovaný obsah. Naneštěstí počítače nedokáží vnímat obraz jako lidé, vnímají totiž obrazy jako sérii binárních informací. Přitom počítače a jejich práce s obrazy by se dala využít v mnoha oborech jako je lékařství nebo doprava. Na základě toho vyplouvá na povrch problém jak spravovat digitální obrázky a efektivně mezi nimi vyhledávat. Prostřednictvím klíčových slov přiřazených k obrázkům se dá problém vyhledávání zjednodušit. Přiřezení klíčových slov probíhá pomocí procesu automatická anotace obrázků, kdy za pomoci trénovací množiny, ze které program natrénujeme, je k obrázku přiřazen jeden nebo více slov které vyjadřují jeho obsah. Automatická anotace obrázku je tedy proces, ve kterém jsou k obrázku automaticky přiřazena metadata, která obsahují klíčová slova (například les, strom). \\

Problematika porovnávání obrázků, po které následuje anotace, se dá rozložit na dva menší problémy. První je získaní informace z trénovací množiny a druhý je jak rozhodovat jestli jsou si obrázky skutečně podobné a nakolik. \\

Cílem práce je navrhnout a implementovat software umožňující automatickou anotaci obrázků. 

POPSAT KONKRÉTNÍ METODU

 
% 
% PRO ANGLICKOU SAZBU JE NUTNÉ ZMĚNIT
% CITAČNÍ STYL!
%
\bibliographystyle{csplainnatkiv}
{\raggedright\small
\bibliography{literatura}
}

\chapter{Joint Equal Contribution JEC}
Existuje skupina základních metod pro anotaci obrázků, která je postavena na hypotéze, že obrázky jsou si podobné s několika obrázky (tzv. sousedy). Přičemž klíčová slova od jednodlivých sousedů jsou posuzována odlišně a to na základě toho o kolik se s testovaným obrázkem liší. 
\section{Příznaky}
Barva a textura jsou považovány za dva nejdůležitější nizko-úrovňové příznaky pro obrázkovou reprezentaci. Nejběžnější barevné deskriptory jsou základem hrubého histogram, který je často využíván s obrázkovým srovnáním a indexovým schématem, primárně z důvodu jejich efektivnosti a snádného výpočtu. Obrázková textura je běžně zachycena s Wavelet vlastností. V části Gabor a Haar wavelets bylo prokázáno, že je velmi efektivní při vytváření rozptýlených diskriminačních obrázkových rysů. Omezení vlivu a sklon k individuálním funkcí, a maximalizování množství obsažených informací vybereme pro prácí pár jednoduchých a snadno vypočítatelných funkcí. 
\section{Vzdálenosti}
asda
\section{Přenesení klíčových slov}
Pro přenesení klíčových slov používáme metodu, kdy přeneseme n klíčových slov k dotazovanému obrázku $\tilde{I}$ od K nejbližších sousedů v trénovací sadě. Mějme $I_{i}, i = 1, ..., K$ ,tyto K nejbližší sousedy seřadíme podle vzrůstající vzdálenosti (tzn. že $I_{1} $ je nejvíce podobný obrázek). Počet klíčových slov k danému $I_{i}$ je označen jako $|I_{i}|$. Dále jsou popsány jednotlivé kroky alogoritmu na přenesení klíčových slov.
\begin{enumerate}
	\item Seřadíme klíčová slova z $I_{1}$ podle jejich frekvence v trénovacích datech.
	\item Z $|I_{1}|$ klíčových slov z $I_{1}$ přeneseme n nejvýše umístěná klíčová slova do dotazovaného $\tilde{I}$. Když $|I_{1}| < n$ pokračujte na krok 3. 
	\item Seřaď klíčová slova sousedů od $I_{2}$ do $I_{K}$ podle dvou faktorů
	\begin{enumerate}
		\item Co výskyt v trénovacích datech s klíčovými slovy přenesených v kroku 2 a
		\item místní frekvence (tj. jak často se vyskytují jako klíčová slova u obrázků $I_{2}$ až $I_{K}$). Vyber nejvyšší rankink $n-|I_{1}|$ klíčových slov převedených do $\tilde{I}$.
	\end{enumerate}
\end{enumerate}

Tento algoritmus pro přenos klíčových slov je poněkud odlišný od algoritmů které se běžně používají. Jeden z běžně užívaných funguje na principu, že klíčová slova jsou vybrána od všech sousedů (se všemi sousedy je zacházeno stejně bez ohledu na to jak jsou danému obrázku podobní), jiný užívaný algoritmus k sousedům přistupuje váženě (každý soused má jinou váhu)a to na základě jejich vzdálenosti od testovaného obrázku. Při testování se ovšem ukázalo, že tyto přímé přístupy přináší horší výsledky v porovnání s námi použitým dvoufaktorovým algoritmem pro přenos klíčových slov. \\
V souhrnu námi použitá metoda je složenina ze svou složeniny obrázkové vzdálenosti měřítku (JEC nebo Lasso) pro nejbližší ranking, kombinuje se s výše popsaným algoritmem na přenášení klíčových slov.
\section{Vyhodnocení}

\chapter{Testovací databáze}
Pro natrénování a následné testování byly data z databází ČTK,ESP a IAPRC.
\chapter{Návrh systému}
\chapter{Použité programové prostředky / Požadavky na software}
Program byl navržen na operační systému Linux. Jako programovací jazyk byl zvolen Python a to z důvodu jeho jednoduchého použití, což je na prototyp, jako je tento velice výhodné na časovou náročnost. Program využívá knihovnu OpenCV 3.1.   
\section{OpenCV}
OpenCV (Open source computer vision) je knihovna vydávána pod licencí BSD a je volně k dispozici jak pro akademické účely, tak pro komerční použití. Je vhodná pro použití v C++, C, Python a Javě. Podporuje operační systémy Windows, Linux, Mac OS, iOS a Android.
\\
Knihovna byla navrhnuta pro výpočetní efektivitu v oblasti počítačového vidění a zpracování obrazu se zaměřením na zpracování obrazu v reálném čase. Z důvodu optimalizace byla napsána v C/C++. 
\\
Knihovnu OpenCV je možné stáhnout na adrese: http://opencv.org/
\\

\chapter{Výsledek}

\end{document}
