%%%%%%%%%%%%%%%%%%%%%%%%%%%%%%%%%%%%%%%%%%%%%%%%%%%%%%%%%%
%
% Vzor pro sazbu kvalifikační práce
%
% Západočeská univerzita v Plzni
% Fakulta aplikovaných věd
% Katedra informatiky a výpočetní techniky
%
% Petr Lobaz, lobaz@kiv.zcu.cz, 2016/03/14
%
%%%%%%%%%%%%%%%%%%%%%%%%%%%%%%%%%%%%%%%%%%%%%%%%%%%%%%%%%%

% Možné jazyky práce: czech, english
% Možné typy práce: BP (bakalářská), DP (diplomová)
\documentclass[czech,BP]{thesiskiv}

% Definujte údaje pro vstupní strany
%
% Jméno a příjmení; kvůli textu prohlášení určete, 
% zda jde o mužské, nebo ženské jméno.
\author{Kateřina Kratochvílová}
\declarationfemale

% Název práce
\title{Automatická anotace\\obrázků}

% 
% Texty abstraktů (anglicky, česky)
%
\abstracttexten{The text of the abstract (in English). It contains the English translation of the thesis title and a short description of the thesis.}

\abstracttextcz{Text abstraktu (česky). Obsahuje krátkou anotaci (cca 10 řádek) v češtině. Budete ji potřebovat i při vyplňování údajů o bakalářské práci ve STAGu. Český i anglický abstrakt by měly být na stejné stránce a měly by si obsahem co možná nejvíce odpovídat (samozřejmě není možný doslovný překlad!).
}

% Na titulní stranu a do textu prohlášení se automaticky vkládá 
% aktuální rok, resp. datum. Můžete je změnit:
%\titlepageyear{2016}
%\declarationdate{1. března 2016}

% Ve zvláštních případech je možné ovlivnit i ostatní texty:
%
%\university{Západočeská univerzita v Plzni}
%\faculty{Fakulta aplikovaných věd}
%\department{Katedra informatiky a výpočetní techniky}
%\subject{Bakalářská práce}
%\titlepagetown{Plzeň}
%\declarationtown{Plzni}

%%%%%%%%%%%%%%%%%%%%%%%%%%%%%%%%%%%%%%%%%%%%%%%%%%%%%%%%%%
%
% DODATEČNÉ BALÍČKY PRO SAZBU
% Jejich užívání či neužívání záleží na libovůli autora 
% práce
%
%%%%%%%%%%%%%%%%%%%%%%%%%%%%%%%%%%%%%%%%%%%%%%%%%%%%%%%%%%

% Zařadit literaturu do obsahu
\usepackage[nottoc,notlot,notlof]{tocbibind}

% Umožňuje vkládání obrázků
\usepackage[pdftex]{graphicx}

% Odkazy v PDF jsou aktivní; navíc se automaticky vkládá
% balíček 'url', který umožňuje např. dělení slov
% uvnitř URL
\usepackage[pdftex]{hyperref}
\hypersetup{colorlinks=true,
  unicode=true,
  linkcolor=black,
  citecolor=black,
  urlcolor=black,
  bookmarksopen=true}

% Při používání citačního stylu csplainnatkiv
% (odvozen z csplainnat, http://repo.or.cz/w/csplainnat.git)
% lze snadno modifikovat vzhled citací v textu
\usepackage[numbers,sort&compress]{natbib}

%%%%%%%%%%%%%%%%%%%%%%%%%%%%%%%%%%%%%%%%%%%%%%%%%%%%%%%%%%
%
% VLASTNÍ TEXT PRÁCE
%
%%%%%%%%%%%%%%%%%%%%%%%%%%%%%%%%%%%%%%%%%%%%%%%%%%%%%%%%%%
\begin{document}
%
\maketitle
\tableofcontents

\chapter{Úvod}

V dnešní době je k dispozici stále více a více obrázků. Avšak vyhledání požadovaného obrázku pro běžné použití je nadlidský úkol. Prostřednictvím klíčových slov přiřazený k obrázku se dá tento problém zjednodušit. Přiřazení klíčových slov probíhá pomocí procesu automatické anotace obrázku, kdy za pomocí trénovací množiny, ze které se program natrénuje, je k obrázku přiřazen jeden nebo více slov které vyjadřují jeho obsah. (Automatická anotace obrázku je proces, ve kterém jsou k obrázku automaticky přiřazena metadata, která obsahují klíčová slova, například (příklady)) Výběr trénovací množiny je v této problematice zásadní. 

Cílem práce je navrhnout a implementovat software umožňující automatickou anotaci obrázků. Popsat konktréní metodu.



K čemu je to dobrý o AIA, spíš doplněk k loňskýmu, možná jaký metody budeme používat a co je cílem práce


Velké množství výzkumu na obrázkové získání byli provedeny v minulých výzkumech. Tradičně výzkum je v této oblasti zaměřen na obsahový základ získávní obrázku. Avšak nedávné výzkumy ukázali že je zde významový rozdíl mezi obsahem základu obrázku a získání semanticky pochopitelných člověkem. Jako výsledek vázkumu v této oblasti je posunuje k překonání mostu semantický mezera je přes automatickou anotaci obrázků (AIA) která extrahuje semantickou vlastnost používá strojové učení techniky. V tomto papíře se zaměříme 
V souboru \texttt{literatura.bib} jsou uvedeny příklady, jak citovat knihu \cite{KnuthAOCP2}, článek v časopisu \cite{Hoare1961}, webovou stránku \cite{Graphics2D}.
 
% 
% PRO ANGLICKOU SAZBU JE NUTNÉ ZMĚNIT
% CITAČNÍ STYL!
%
\bibliographystyle{csplainnatkiv}
{\raggedright\small
\bibliography{literatura}
}

\chapter{Znaky a vzdálenosti}
\section{Lorem ipsum dolor sit amet}
...
\subsection{Nunc et ante}

 Existuje skupina základních metod pro obrázkovou anotaci, která je postavena na hypotéze, že na základě podobnosti vzhledu obrázku jsou podílově přiřazena klíčová slova. K obrázkové anotaci se přistupuje jako k procesu přenášení klíčových slov od nejbližších sousedů. Struktura sousedů je konstruovaná použitím jednoduchých low-level obrázkových příznaků.

Barva a textura jsou považovány za dva nejdůležitější nizko-úrovňové příznaky pro obrázkovou reprezentaci. Nejběžnější barevné deskriptory jsou základem hrubého histogram, který je často využíván s obrázkovým srovnáním a indexovým schématem, primárně z důvodu jejich efektivnosti a snádného výpočtu. Obrázková textura je běžně zachycena s Wavelet vlastností. V části Gabor a Haar wavelets bylo prokázáno, že je velmi efektivní při vytváření rozptýlených diskriminačních obrázkových rysů. Omezení vlivu a sklon k individuálním funkcí, a maximalizování množství obsažených informací vybereme pro prácí pár jednoduchých a snadno vypočítatelných funkcí. 


\chapter{Přenesení klíčových slov}




\end{document}
